%------------------------------------------------------------------------------------------------------------------------------%
%                                                           Packages                                                           %
%------------------------------------------------------------------------------------------------------------------------------%
\usepackage[margin=2.0cm,a5paper]{geometry}
\usepackage{fontspec}
\usepackage{graphicx}
\usepackage[greek,english,portuges]{babel}   % last becomes the active one
\usepackage{indentfirst}
%------------------------------------------------------------------------------------------------------------------------------%
%                                                            Fonts                                                             %
%------------------------------------------------------------------------------------------------------------------------------%
\defaultfontfeatures{
    Ligatures               = TeX,
    Numbers                 = {Monospaced, Uppercase}, % Lining, Proportional, OldStyle
}

\setmainfont{QTCaslan}[
    Extension               = .otf,
    ItalicFont              = *-Italic,
    BoldFont                = *-Bold,
    BoldItalicFont          = *-BoldItalic,
    SlantedFeatures         = {FakeSlant=0.2},
]

%%%\setmainfont{QTBodini}[
%%%    Extension               = .otf,
%%%    ItalicFont              = *-Italic,
%%%    BoldFont                = *-Bold,
%%%    BoldItalicFont          = *-Italic,
%%%    BoldItalicFeatures      = {FakeBold=3.0},
%%%    SlantedFeatures         = {FakeSlant=0.2},
%%%]

%%%\setmainfont{QTBasker}[
%%%    Extension               = .otf,
%%%    ItalicFont              = *-Italic,
%%%    BoldFont                = *-Bold,
%%%    BoldItalicFont          = *-Italic,
%%%    BoldItalicFeatures      = {FakeBold=5.0},
%%%    SlantedFeatures         = {FakeSlant=0.2},
%%%]

%%%\setmainfont{CormorantGaramond}[
%%%    Path                    = /usr/share/fonts/TTF/,
%%%    Extension               = .ttf,
%%%    UprightFont             = *-Medium,
%%%    ItalicFont              = *-MediumItalic,
%%%    BoldFont                = *-Bold,
%%%    BoldItalicFont          = *-BoldItalic,
%%%    BoldFeatures            = {FakeBold=1.5},
%%%    BoldItalicFeatures      = {FakeBold=1.5},
%%%    SlantedFeatures         = {FakeSlant=0.2},
%%%]
\setsansfont{FiraSans-Regular}[
    Path                    = /usr/share/fonts/TTF/,
    Extension               = .ttf,
    ItalicFont              = FiraSans-Italic,
    BoldFont                = FiraSans-ExtraBold,
    BoldItalicFont          = FiraSans-ExtraBoldItalic,
    Scale                   = MatchLowercase,
    SlantedFeatures         = {FakeSlant=0.2},
]
\setmonofont{JuliaMono}[
    Path            = /usr/share/fonts/TTF/,
    Extension       = .ttf,
    UprightFont     = *-Medium,
    ItalicFont      = *-MediumItalic,
    BoldFont        = *-Bold,
    BoldItalicFont  = *-BoldItalic,
    Scale           = MatchLowercase,
]
\newfontfamily\cormorantinfantfamily{CormorantInfant-Medium}[
    Path                    = /usr/share/fonts/TTF/,
    Extension               = .ttf,
    ItalicFont              = CormorantInfant-MediumItalic,
    BoldFont                = CormorantInfant-Bold,
    BoldItalicFont          = CormorantInfant-BoldItalic,
    BoldFeatures            = {FakeBold=1.5},
    BoldItalicFeatures      = {FakeBold=1.5},
    SlantedFeatures         = {FakeSlant=0.2},
    Scale                   = MatchLowercase,
]
\newfontfeature{ScriptureColor}{color=C00000FF}
%------------------------------------------------------------------------------------------------------------------------------%
%                                                           Commands                                                           %
%------------------------------------------------------------------------------------------------------------------------------%
\newcommand{\XXX}[1]{\relax}
%------------------------------------------------------------------------------------------------------------------------------%
\newcommand{\YA}{%
    \mbox{%
        Y\makebox[0pt][l]{\hspace{-0.178em}\raisebox{-0.00ex}{\scalebox{0.30}{E}}}%
        H\makebox[0pt][l]{\hspace{-0.010em}\raisebox{-0.00ex}{\scalebox{0.30}{O}}}%
        W\makebox[0pt][l]{\hspace{-0.245em}\raisebox{-0.00ex}{\scalebox{0.30}{A}}}%
        H%
    }%
}
%------------------------------------------------------------------------------------------------------------------------------%
\newcommand{\ver}[1]{%
    \raisebox{0.50ex}{%
        \scalebox{1.1}{%
            \pmb{\textbf{{#1}}}%
        }%
    }%
}
%------------------------------------------------------------------------------------------------------------------------------%
\newcommand{\bLineQuote}[3]{%
    ``\textrm{\addfontfeature{ScriptureColor}{#1}}'' {#2} ({#3})~\cite{{#3}}%
}
\newcommand{\bBlockQuoteAft}[3]{%
    \par\noindent\hspace*{0.1\linewidth}%
    \begin{minipage}{0.8\linewidth}\raggedleft%
        \noindent ``\textrm{\addfontfeature{ScriptureColor}{#1}}'' ---~{#2}~({#3})~\cite{{#3}}
    \end{minipage}%
    \vspace*{1.0\baselineskip}%
    \linebreak%
}
\newcommand{\bBlockQuote}[3]{%
    \vspace*{1.0\baselineskip}%
    \bBlockQuoteAft{{#1}}{{#2}}{{#3}}
}
%------------------------------------------------------------------------------------------------------------------------------%
%                                                           Metadata                                                           %
%------------------------------------------------------------------------------------------------------------------------------%
\makeatletter
\immediate\write18{datelog.pt > \jobname.info}
\makeatother
%-------------------------------------------------------------------------------------------------------------------------------
% Article title
\title{\textbf{As Origens da Grande Tribulação}}
\author{C.~Naaktgeboren\thanks{\texttt{C.$\,$Naaktgeboren <bibliashare\textcircled{a}gmail.com>}},}
\date{{\scriptsize\texttt{Compilado em \input{\jobname.info} - Revisão 0
}}}
%-------------------------------------------------------------------------------------------------------------------------------
