%------------------------------------------------------------------------------------------------------------------------------%
%                                                          Title Page                                                          %
%------------------------------------------------------------------------------------------------------------------------------%

\thispagestyle{empty} % Removes page numbering from the first page
\flushbottom % Makes all text pages the same height
\maketitle % Print the title and abstract box


%------------------------------------------------------------------------------------------------------------------------------%
%                                                           License                                                            %
%------------------------------------------------------------------------------------------------------------------------------%

\section*{License}

    \scriptsize\noindent%
    \begin{minipage}{\columnwidth}
        \centering\tt
        \includegraphics[height=6.0mm]{cc/by-nc.pdf}\\[0.5\smallskipamount]
        {\scriptsize\url{https://creativecommons.org/licenses/by-nc/4.0/}}
    \end{minipage}
    \normalsize


%------------------------------------------------------------------------------------------------------------------------------%
%                                                      Table Of Contents                                                       %
%------------------------------------------------------------------------------------------------------------------------------%

\tableofcontents


%------------------------------------------------------------------------------------------------------------------------------%
%                                                         Introduction                                                         %
%------------------------------------------------------------------------------------------------------------------------------%

\section{Introdução}

    Este estudo aborda o assunto da ``tempo de angústia para Jacó,'' profetizado pelos profetas Jeremias e Daniel, e referido em
    profecia dada pelo Senhor Jesus como o período de ``grande  tribulação,''  passagens  das  Escrituras  que  lêem,  em  ordem
    cronológica\footnote{Uma vez que Jeremias é citado por Daniel, conforme Dn~9.2, e que Daniel é citado pelo Senhor Jesus,
    conforme Mt~24.15, a ordem cronológica é: Jeremias -- Daniel -- Senhor Jesus.}:

        \bBlockQuote{% !j -i12 116
            Ah! Que grande é aquele dia, e não há outro semelhante! É tempo de angústia para Jacó; ele, porém,  será
            livre dela.}{Jr~30.7}{ARA}

        \bBlockQuoteAft{% !j -i12 116
            e haverá tempo de angústia, qual nunca houve, desde que houve nação até àquele tempo;}{Dn~12.1}{ARA}

        \bBlockQuoteAft{% !j -i12 116
            porque nesse tempo haverá grande tribulação, como desde o princípio do mundo até agora não tem havido  e
            nem haverá jamais.}{Mt~24.21}{ARA}

    Em particular, o tópico é abordado em sua \emph{suposta} relação com a igreja, na questão de se a igreja passa  ou  não  por
    tal período; e, se passa; em qual fração de sua duração. Algumas das linhas de  interpretação  profética  existentes  e  que
    diferem nesta questão, podem ser arranjadas em ordem crescente de \emph{suposta}  participação  da  igreja,  desde:  nenhuma
    participação, no caso do pré-tribulacionismo; participação até a metade, no caso do meso-tribulacionismo;  até  participação
    completa, no caso do pós-tribulacionismo; entre outras. As designações `pré-', `meso-' e `pós-' --- as quais traduzem-se por
    `antes', `intermediário' e `após' --- referem-se ao posicionamento temporal do \emph{arrebatamento da igreja} em relação  ao
    tempo da \emph{grande tribulação} --- sendo o arrebatamento o evento profético que retira a igreja deste mundo para que esta
    esteja \bLineQuote{para sempre com o Senhor}{1Ts~4.17}{ARA}.

    Não obstante as Escrituras exortarem a que a igreja tenha um só pensamento, para a completa alegria: \bLineQuote{completai a
    minha alegria, de modo que \emph{penseis a mesma coisa}, tenhais o  mesmo  amor,  sejais  unidos  de  alma,  tendo  o  mesmo
    sentimento.}{Fp~2.2}{ARA}; vemos, em nosso meio, defensores de cada uma das visões elencadas,  cada  qual  com  seu  rol  de
    textos e estratégias de interpretação.

    Tal estado de coisas é lamentável por múltiplas razões, incluindo: (i)~não se cumpre a exortação de Fp~2.2, para  cujo  caso
    reserva-se a esperança do verso 3.15 da mesma Epístola: \bLineQuote{todos os que somos  aperfeiçoados  tenhamos  esse  mesmo
    modo de pensar; e, se em alguma coisa pensais de outro modo, Deus também vos revelará isso.}{Fp~3.15}{A21}; e  (ii)~corre-se
    o risco imediato de transmitir, voluntariamente ou não, a mensagem de que a Bíblia não seria coesa, ou  pior,  que  conteria
    contradições. Porém o texto citado de Fp~3.15 responde, de imediato, à tais fontes de lamento, atribuindo o pensar igual não
    apenas ao ``ser aperfeiçoado,'' mas eminentemente ao \emph{receber revelação de Deus}; e assim,  identificando  a  fonte  do
    problema no interpretar textos não segundo Deus; e não nas Escrituras propriamente ditas!

    Além disso, a necessidade de revelação \emph{divina} em Fp~3.15, mostra que unidade de  pensamento  na  igreja  jamais  será
    alcançado enquanto os demais tiverem que pensar ``como eu'' --- do ponto de vista de alguém; mas sim quando  todos  pensarem
    \emph{segundo Deus} --- haja vista que sua \emph{inspiração Divina} e \emph{inerrância} são axiomáticas!

    A busca por uma interpretação de profecia \emph{segundo Deus} certamente nos convida a analisar cada verso,  cada  sentença,
    cada expressão \emph{à luz das Escrituras}, assim como manter em consideração aspectos do próprio \emph{caráter de Deus}.  A
    interpretação de profecias passa a ser um \emph{projeto de caminhada e vida com Deus, sempre à luz da Sua Palavra}, afinal o
    Espírito Santo afirma, pelo salmista: \bLineQuote{Compreendo mais do que todos os  meus  mestres,  porque  medito  nos  teus
    testemunhos.}{Sl~119.99}{ARA}, indicando que a meditação na Palavra, e,  por  extensão,  a  interpretação  da  Palavra  pela
    Palavra leva nossa compreensão  mais  além  daquilo  que  alcançam  mestres  formados  por  expedientes  humanos,  incluindo
    eminentemente a escolaridade acadêmica.

    Temos exemplos disso no próprio Verbo encarnado:

        \bBlockQuote{% !j -i12 116
            Terminados os dias da festa, ao regressarem, permaneceu o menino Jesus em Jerusalém, sem que seus pais o
            soubessem. [...] Três dias depois, o acharam no templo, assentado no meio  dos  doutores,  ouvindo-os  e
            interrogando-os. E todos os que o ouviam muito se admiravam da sua inteligência e  das  suas  respostas.
            [...]   E   crescia   Jesus   em   sabedoria,   estatura   e   graça,   diante    de    Deus    e    dos
            homens.}{Lc~2.43,46,47,52}{ARA}

    O texto evidencia a sabedoria e graça vindas do alto, operando na vida do \emph{menino} Jesus, com absoluta superioridade em
    relação ao expediente humano da escolaridade, porquanto o menino de doze anos ouvia e  interrogava  doutores  (da  Lei),  os
    quais ``muito se admiravam da sua inteligência e das suas respostas.''

    Ainda mais:

        \bBlockQuote{% !j -i12 116
            Chegando o sábado, passou a ensinar na sinagoga; e muitos, ouvindo-o, se  maravilhavam,  dizendo:  Donde
            vêm a este estas coisas? Que sabedoria é esta que lhe foi dada? E como se fazem tais maravilhas por suas
            mãos? Não é este o carpinteiro, filho de Maria, irmão de Tiago, José, Judas e Simão? E  não  vivem  aqui
            entre nós suas irmãs? E escandalizavam-se nele.}{Mc~6.2,3}{ARA}

    A falta de notoriedade imbutida nas palavras ``o carpinteiro,'' filho de conhecidos e cujas irmãs vivem entre nós é patente,
    assim como a reação natural: ``escandalizavam-se nele.''

    E ainda, com relação aos Apóstolos:

        \bBlockQuote{% !j -i12 116
            Ao verem a intrepidez de Pedro e João, sabendo que eram homens iletrados  e  incultos,  admiraram-se;  e
            reconheceram que haviam eles estado com Jesus.}{At~4.13}{ARA}

    Nesta última citação, a falta de preparo acadêmico é especialmente ressaltada nos termos ``iletrados e incultos,'' ao  passo
    que o convívio com a Palavra (encarnada) foi deduzido logo na sequência: ``reconheceram que haviam eles estado com Jesus.''




    De outro ponto de vista, desta vez prático e não doutrinal; as diferentes visões com relação à  participação  da  igreja  na
    grande tribulação possuem consequências práticas, a exemplo da (a)~eventual necessidade  de  preparações;  do  (b)~molde  de
    expectativas com relação aos dias finais --- os quais, inevitavelmente, farão parte de um \emph{futuro próximo e  iminente};
    e também (c)~do trato de Deus para com a igreja e suas profundas implicações;  do  (d)~caráter  de  Deus  e  suas  profundas
    implicações.

    Em seu estudo percorrendo todos os livros da Bíblia, o Dr.~J. Vernon McGee chega ao Apocalipse de João  ---  o  único  livro
    profético do Novo Testamento --- identificando que o livro dá consumação a uma série de assuntos proféticos vindos de várias
    outras partes das  Escrituras~\cite{ca1980-McGeeJV-49Rev}.  Um  dos  assuntos  proféticos  identificados  é  o  da  ``Grande
    Tribulação,'' o qual, segundo McGee, tem sua origem no Antigo Testamento; na Lei; especificamente em Deuteronômio 4.30,~31.

    Tal metodologia de estudo me parece ser a mais desejável, bem como a mais apropriada, uma vez que:  (i)~utilizará  a  Bíblia
    para explicar a própria Bíblia; e também (ii)~fará isso na \emph{ordem} na qual a revelação aconteceu na história.

    Este estudo empregará a \emph{metodologia} adotada por McGee, porém, de forma completamente independente  de  seus  estudos,
    objetivando \emph{descobrir o que é  ensinado  nas  Escrituras  sobre  o  assunto},  não  desejando  uma  validação  de  uma
    pré-determinada visão de mundo, porém deixando a Escritura (Deus) falar e colhendo  os  resultados  da  desejada  coesão  (e
    correção!) doutrinária.


%------------------------------------------------------------------------------------------------------------------------------%
%                                                         Conclusions                                                          %
%------------------------------------------------------------------------------------------------------------------------------%

\section{Conclusão}

    Testes.

%-------------------------------------------------------------------------------------------------------------------------------
