%------------------------------------------------------------------------------------------------------------------------------%
%                                                          Title Page                                                          %
%------------------------------------------------------------------------------------------------------------------------------%

\thispagestyle{empty} % Removes page numbering from the first page
\flushbottom % Makes all text pages the same height
\maketitle % Print the title and abstract box


%------------------------------------------------------------------------------------------------------------------------------%
%                                                           License                                                            %
%------------------------------------------------------------------------------------------------------------------------------%

\section*{License}

    \scriptsize\noindent%
    \begin{minipage}{\columnwidth}
        \centering\tt
        \includegraphics[height=6.0mm]{cc/by-nc.pdf}\\[0.5\smallskipamount]
        {\scriptsize\url{https://creativecommons.org/licenses/by-nc/4.0/}}
    \end{minipage}
    \normalsize


%------------------------------------------------------------------------------------------------------------------------------%
%                                                      Table Of Contents                                                       %
%------------------------------------------------------------------------------------------------------------------------------%

\tableofcontents


%------------------------------------------------------------------------------------------------------------------------------%
%                                                         Introduction                                                         %
%------------------------------------------------------------------------------------------------------------------------------%

\section{Introdução}

    O assunto da ``angústia de Jacó,'' mencionado pelos profetas Jeremias e Daniel, e referido pelo Senhor Jesus como o período
    de ``grande tribulação,'' citamos:

        \bBlockQuote{% !j -i12 116
            Ah! Que grande é aquele dia, e não há outro semelhante! É tempo de angústia para Jacó; ele, porém,  será
            livre dela.}{Jr~30.7}{ARA}

        \bBlockQuoteAft{% !j -i12 116
            e haverá tempo de angústia, qual nunca houve, desde que houve nação até àquele tempo;}{Dn~12.1}{ARA}

        \bBlockQuoteAft{% !j -i12 116
            porque nesse tempo haverá grande tribulação, como desde o princípio do mundo até agora não tem havido  e
            nem haverá jamais.}{Mt~24.21}{ARA}

    O assunto da ``Grande Tribulação,'' mencionado na visão de Daniel: \bLineQuote{[...] e haverá tempo de angústia, qual  nunca
    houve, desde que houve nação até àquele tempo; [...]}{Dn~12.1}{ARA} e também pelo  Senhor  Jesus:  \bLineQuote{porque  nesse
    tempo  haverá  grande  tribulação,  como  desde  o  princípio  do  mundo  até  agora   não   tem   havido   e   nem   haverá
    jamais.}{Mt~24.21}{ARA}; vem sendo, de acordo com as minhas observações, objeto de debate no meio cristão, principalmente no
    tocante à relação da igreja com este período profético, a saber, se a igreja deverá, ou não passar por  esse  tempo;  e,  se
    passar, o fará em parte ou na sua totalidade.

    Para  os  que  gostam  de  classificações,  as  opções  enunciadas  correspondem   à   visões   (i)~pré-,   (ii)~meso-,   ou
    (iii)~pós-tribulacionistas, para citar as principais, e as nomenclaturas advém do posicionamento do arrebatamento da  igreja
    em relação ao tempo da grande tribulação, a saber: antes, no meio, ou no final dela, respectivamente.

    Não obstante as Escrituras exortarem a que a igreja tenha um só pensamento, para a completa alegria: \bLineQuote{completai a
    minha alegria, de modo que \emph{penseis a mesma coisa}, tenhais o  mesmo  amor,  sejais  unidos  de  alma,  tendo  o  mesmo
    sentimento.}{Fp~2.2}{ARA}; vemos, em nosso meio, defensores de cada uma das visões elencadas,  cada  qual  com  seu  rol  de
    textos e estratégias de interpretação.

    Considero triste tal estado de coisas, por múltiplas razões: (i)~não  atinge-se  a  exortação  de  Fp~2.2,  para  cujo  caso
    reserva-se a esperança do verso 3.15: \bLineQuote{Por isso, todos os que somos aperfeiçoados tenhamos  esse  mesmo  modo  de
    pensar; e, se em alguma coisa pensais de outro modo, Deus também vos revelará isso.}{Fp~3.15}{A21}; e (ii)~corre-se o  risco
    imediato de transmitir, voluntariamente ou não, a  mensagem  de  que  a  Bíblia  não  seja  coesa,  ou  pior,  que  contenha
    contradições.

    O problema não está nas Escrituras em si --- haja vista que sua inspiração Divina e inerrância são axiomáticas  ---  mas  na
    trajetória de crescimento na fé, inerente a cada cristão; bem como em posturas evitáveis como a defesa de visões;  ao  invés
    de uma busca pelo que é, de fato, ensinado nas Escrituras; afinal, o que aproveita alguém engajado na defesa de erros?

    De outro ponto de vista, desta vez prático e não doutrinal; as diferentes visões com relação à  participação  da  igreja  na
    grande tribulação possuem consequências práticas, a exemplo da (a)~eventual necessidade  de  preparações,  e  também  (b)~do
    trato de Deus para com a igreja e suas profundas implicações.

    Em seu estudo percorrendo todos os livros da Bíblia, o Dr.~J. Vernon McGee chega ao Apocalipse de João  ---  o  único  livro
    profético do Novo Testamento --- identificando que o livro dá consumação a uma série de assuntos proféticos vindos de várias
    outras partes das  Escrituras~\cite{ca1980-McGeeJV-49Rev}.  Um  dos  assuntos  proféticos  identificados  é  o  da  ``Grande
    Tribulação,'' o qual, segundo McGee, tem sua origem no Antigo Testamento; na Lei; especificamente em Deuteronômio 4.30,~31.

    Tal metodologia de estudo me parece ser a mais desejável, bem como a mais apropriada, uma vez que: (i)~utilizará a Bíblia
    para explicar a própria Bíblia; e também (ii)~fará isso na \emph{ordem} na qual a revelação aconteceu na história.

    Este estudo empregará a \emph{metodologia} adotada por McGee, porém, de forma completamente independente  de  seus  estudos,
    objetivando \emph{descobrir o que é  ensinado  nas  Escrituras  sobre  o  assunto},  não  desejando  uma  validação  de  uma
    pré-determinada visão de mundo, porém deixando a Escritura (Deus) falar e colhendo  os  resultados  da  desejada  coesão  (e
    correção!) doutrinária.


%------------------------------------------------------------------------------------------------------------------------------%
%                                                         Conclusions                                                          %
%------------------------------------------------------------------------------------------------------------------------------%

\section{Conclusão}

    Testes.

%-------------------------------------------------------------------------------------------------------------------------------
