% !center 124 | frame 124 -f'\%-\% '
%------------------------------------------------------------------------------------------------------------------------------%
%                                                          Title Page                                                          %
%------------------------------------------------------------------------------------------------------------------------------%
\thispagestyle{empty} % Removes page numbering from the first page
\flushbottom % Makes all text pages the same height
\maketitle % Print the title and abstract box


% !center 124 | frame 124 -f'\%-\% '
%------------------------------------------------------------------------------------------------------------------------------%
%                                                           Abstract                                                           %
%------------------------------------------------------------------------------------------------------------------------------%
\begin{abstract}
    Aqui vai o resumo.
\end{abstract}


% !center 124 | frame 124 -f'\%-\% '
%------------------------------------------------------------------------------------------------------------------------------%
%                                                           License                                                            %
%------------------------------------------------------------------------------------------------------------------------------%
\section*{Licença}

    \scriptsize\noindent%
    \begin{minipage}{\columnwidth}
        \centering\tt
        \includegraphics[height=6.0mm]{cc/by-nc.pdf}\\[0.5\smallskipamount]
        {\scriptsize\texttt{https://creativecommons.org/licenses/by-nc/4.0/}}
    \end{minipage}
    \normalsize


% !center 124 | frame 124 -f'\%-\% '
%------------------------------------------------------------------------------------------------------------------------------%
%                                                      Table Of Contents                                                       %
%------------------------------------------------------------------------------------------------------------------------------%
\tableofcontents


% !center 124 | frame 124 -f'\%-\% '
%------------------------------------------------------------------------------------------------------------------------------%
%                                                     Measure Adjustments                                                      %
%------------------------------------------------------------------------------------------------------------------------------%
\setlength{\parskip}{0.5\baselineskip}


% !center 124 | frame 124 -f'\%-\% '
%------------------------------------------------------------------------------------------------------------------------------%
%                                                         Introduction                                                         %
%------------------------------------------------------------------------------------------------------------------------------%

\section{Introdução}

    Este estudo aborda o assunto da ``\emph{grande tribulação},'' enunciada pelo Senhor Jesus no Monte das Oliveiras:

        \bBlockQuote{% !j -i12 116
            porque nesse tempo haverá \textbf{grande tribulação}, como desde o princípio do mundo até agora não  tem
            havido e nem haverá jamais.}{Mt~24.21}{ARA}

    Também o profeta Daniel, assim chamado pelo próprio Senhor Jesus\footnote{\bLineQuote{Quando, pois, virdes o  abominável  da
    desolação de que falou o \textit{profeta Daniel}, no lugar santo}{Mt~24.15}{ARA}.}, falou sobre o assunto da tribulação:

        \bBlockQuote{% !j -i12 116
            Nesse tempo, se levantará Miguel, o grande príncipe, o  defensor  dos  filhos  do  teu  povo,  e  haverá
            \textbf{tempo de angústia}, qual nunca houve, desde que houve  nação  até  àquele  tempo;  mas,  naquele
            tempo, será salvo o teu povo, todo aquele que for achado inscrito no livro.}{Dn~12.1}{ARA}

    Ambas as descrições são de \emph{angústia ou tribulação sem precedentes}; por isso sabemos que ambos o profeta  Daniel  e  o
    Senhor Jesus estão referindo-se ao \emph{mesmo período profético}.

    Para o tempo profetizado em Dn~12.1, temos o levante do \bQuote{defensor dos filhos do teu povo},  assim  como  \bQuote{será
    salvo o teu povo}; ora, o ``povo de Daniel,'' segundo as Escrituras, é  \emph{Israel},  conforme:  \bLineQuote{meu  povo  de
    Israel}{de Dn~9.20}{ARA}.

    Ora, como Israel é Jacó, sabemos que o profeta Jeremias também falou da tribulação, em termos de \bQuote{tempo  de  angústia
    para Jacó}:

        \bBlockQuote{% !j -i12 116
            Ah! Que grande é aquele dia, e não há outro semelhante! É \textbf{tempo de  angústia}  para  Jacó;  ele,
            porém, será livre dela.}{Jr~30.7}{ARA}

    Em particular, o tópico da grande tribulação é abordado em sua \emph{eventual} relação com a igreja, nas questões de  (i)~se
    a aludida relação existe e, caso afirmativo, (ii)~qual seja a  relação,  de  modo  a  concluir,  à  partir  das  Escrituras,
    aplicações práticas a exemplo de se a igreja também passa ou não por tal período; e, se também passa; em qual fração de  sua
    duração.

    %---------------------------------------------------------------------------------------------------------------------------
    \subsection{Objetivo Geral}

    Visto que para a igreja existe a promessa de seu \emph{arrebatamento}, sendo este o evento profético  que  retira  a  igreja
    deste mundo a fim de que ela esteja \bLineQuote{para sempre com o Senhor}{1Ts~4.17}{ARA}, o  estudo  proposto  traduz-se  no
    objetivo de \emph{posicionar o arrebatamento da igreja em relação ao período da grande tribulação}.

    %---------------------------------------------------------------------------------------------------------------------------
    \subsection{Axiomas}

    O assunto já delineado será estudado com base nos seguintes axiomas:

    \begin{enumerate}

        \item\label{ax:Deus.exis} Há um só Deus;

        \item\label{ax:Escr.pala} As Escrituras Bíblicas são Palavra deste Deus;

        \item\label{ax:Deus.verd} As Escrituras Bíblicas são verdade.

    \end{enumerate}

    Entende-se por ``Escrituras Bíblicas'' o conjunto coeso de 66 livros, composto pelos 39 livros da Bíblia Hebraica e pelos 27
    livros do Novo Testamento Cristão.

% !center 124 | frame 124 -f'\%-\% '
%------------------------------------------------------------------------------------------------------------------------------%
%                                             Estudo de Profecias ``Segundo Deus''                                             %
%------------------------------------------------------------------------------------------------------------------------------%

\section{Estudo de Profecias ``Segundo Deus''}

    %---------------------------------------------------------------------------------------------------------------------------
    \subsection{Da Unicidade da Realidade do Princípio ao Fim}

    As Escrituras \emph{sempre são assertivas} em relação à \emph{realidade} e à \emph{história}, a exemplo de:

        \bBlockQuote{% !j -i12 116
            E fez Deus a expansão e fez separação entre as águas que estavam debaixo da  expansão  e  as  águas  que
            estavam sobre a expansão. \textbf{E assim foi}.}{Gn~1.7}{ARC}

    A sentença \bQuote{E assim foi,} indica uma \textbf{realidade e história únicas} --- ``assim,'' e não de outra forma ---  de
    modo que o espaço-tempo dos \bQuote{céus e terra} possui \textbf{unicidade}, significando uma  \emph{única  realidade},  uma
    \emph{única história} e um \emph{único futuro}.

    Corrobora com a revelação da unicidade da realidade do princípio ao fim, a declaração Divina:
        
        \bBlockQuote{% !j -i12 116
            Lembrai-vos das coisas passadas desde a antiguidade: que \textbf{eu sou Deus, e não há outro Deus},  não
            há outro semelhante a mim; que \textbf{anuncio o fim desde o  princípio}  e,  desde  a  antiguidade,  as
            coisas que ainda não sucederam; que digo: \textbf{o meu conselho será firme},  e  \textbf{farei  toda  a
            minha vontade};}{Is~46.9,10}{ARC}

    Portanto, a capacidade de anunciar, \textbf{acertadamente} \bQuote{coisas que ainda não sucederam} é  um  \emph{atributo  de
    Deus, que o distingue de todos os demais}, conforme o: \bQuote{não há outro semelhante a mim}. Ainda, o  que  Deus  anuncia,
    pela sua Palavra, é \bQuote{o fim desde o princípio} --- note: ``o fim,'' e não uma multiplicidade de `possíveis' fins.

    Está provado, então, a \emph{unicidade da realidade do princípio  ao  fim}:  uma  \emph{única  realidade},  uma  \emph{única
    história} e um \emph{único futuro}.

    %---------------------------------------------------------------------------------------------------------------------------
    \subsection{Da Verdade Das Profecias Divinas}

    O Senhor Deus, ao reiterar seus atributos a Judá, por meio do profeta Isaías, o faz de forma \emph{taxativa}:

        \bBlockQuote{% !j -i12 116
            Porque assim diz o Senhor, que \textbf{criou os céus}, o Deus que \textbf{formou a terra}, que \textbf{a
            fez e a estabeleceu}; que não a criou para ser um caos, mas para ser habitada: \textbf{Eu sou o  Senhor,
            e não há outro}.}{Is~45.18}{ARA}

    Sabemos, pela Carta aos Romanos, que \bLineQuote{os atributos invisíveis de Deus, assim o seu eterno poder,  como  também  a
    sua própria divindade, claramente se reconhecem, desde o princípio do mundo, sendo percebidos por meio das coisas que  foram
    criadas.}{Rm~1.20}{ARA}, de sorte que hoje sabemos que Deus está a evocar  Seus  atributos  de  \bQuote{eterno  poder,  como
    também a sua própria divindade} ao declarar-se Autor de céus e terra, quando falou por meio do profeta Isaías.

    Ainda, Deus segue, por meio do profeta:

        \bBlockQuote{% !j -i12 116
            Não falei em \textbf{segredo}, nem em lugar algum de \textbf{trevas} da terra; não disse à  descendência
            de  Jacó:  Buscai-me  em  vão;  \textbf{eu,  o  Senhor,  falo   a   verdade   e   proclamo   o   que   é
            direito}.}{Is~45.19}{ARA}

    Aqui é acrescentado que a revelação de Deus não foi secreta e com o bendito testemunho: \bQuote{eu, o Senhor, \textbf{falo a
    verdade e proclamo o que é direito}}.

    Assim, está diretamente declarado nas Escrituras que as proclamações de Deus por intermédio de seus profetas  ---  a  saber,
    \emph{todas as profecias} --- \emph{são verdade e direito}.

    %---------------------------------------------------------------------------------------------------------------------------
    \subsection{Profecias Divinas Como Promessas}

        \bBlockQuote{% !j -i12 116
            E assim, depois de esperar com paciência, obteve Abraão a promessa.}{Hb~6.15}{ARA}
        
    A veracidade das profecias divinas implica em certeza de seu cumprimento, portanto \emph{as profecias divinas são  promessas
    divinas}, mas quais pode-se esperar --- \bLineQuote{É o caso de Abraão, que creu em Deus,  e  isso  lhe  foi  imputado  para
    justiça.}{Gl~3.6}{ARA}.

    %---------------------------------------------------------------------------------------------------------------------------
    \subsection{Da Verificabilidade Das Profecias Divinas}

    Sendo a realidade única e as profecias sempre verdadeiras; com a passagem do tempo, aquilo que antes era futuro, a saber, as
    \bQuote{coisas  que  ainda  não  sucederam}{Is~46.10}{ARC},  uma  vez  chegado  seu  tempo  e  cumpridas,  podem  ser  assim
    testemunhadas, ou verificadas, pelos homens. Tais exercícios de constatação são frequentemente registrados nas Escrituras:

        \bBlockQuote{% !j -i12 116
            \textbf{Nenhuma promessa falhou} de todas as \textbf{boas palavras  que  o  Senhor  falara}  à  casa  de
            Israel; \textbf{tudo se cumpriu}.}{Js~21.45}{ARA}
 
    \bQuote{Nenhuma promessa falhou} / \bQuote{tudo se cumpriu.} --- as profecias divinas são  verificáveis  a  seu  tempo.  Que
    maravilha!

    %---------------------------------------------------------------------------------------------------------------------------
    \subsection{Deus Vela Sobre Sua Palavra Para a Cumprir}

    As Escrituras frequentemente explicam que certas coisas vieram a acontecer  com  o  propósito  específico  de  \emph{cumprir
    profecia}, de \emph{cumprir o que está escrito}:

        \bBlockQuote{% !j -i12 116
            Tudo isto, porém, aconteceu para que se cumprissem as Escrituras dos profetas. Então, os discípulos todos,
            deixando-o, fugiram.}{Mt~26.56}{ARA}

    Eminentemente, temos a visão da vara de amendoeira, dada a Jeremias: \bLineQuote{Veio ainda a palavra do Senhor, dizendo:
    Que vês tu, Jeremias? Respondi: vejo uma vara de amendoeira.}{Jr~1.11}{ARA}, e a resposta divina foi:

        \bBlockQuote{% !j -i12 116
            Disse-me  o  Senhor:  Viste   bem,   porque   \textbf{eu   velo   sobre   a   minha   palavra   para   a
            cumprir.}}{Jr~1.12}{ARA}

    Note-se que `velar' significa: ``permanecer de vigia, de sentinela''~\cite{2009-Houaiss+Franco-Objetiva}. Assim, o Deus  que
    está \bLineQuote{sustentando todas as coisas pela palavra do seu  poder}{Hb~1.3}{ARA},  que  \bQuote{é  antes  de  todas  as
    coisas} e no qual \bLineQuote{tudo subsiste}{Cl~1.17}{ARA}, permanece de sentinela para \textbf{cumprir} Sua Palavra!

    %---------------------------------------------------------------------------------------------------------------------------
    \subsection{Cumprimento Literal ou Alegórico?}

    Para que não haja qualquer dúvida sobre a firmeza do propósito Divino no cumprimento fiel de  suas  promessas  e  profecias,
    tem-se, no Livro de Deuteronômio --- portanto na Lei, a profecia da vinda do  profeta  em  cuja  boca  Deus  colocaria  Suas
    Palavras:

        \bBlockQuote{% !j -i12 116
            Suscitar-lhes-ei um profeta do meio de seus irmãos, semelhante a  ti,  em  cuja  boca  porei  as  minhas
            palavras, e ele lhes falará tudo o que eu lhe ordenar.}{Dt~18.18}{ARA}

    A profecia é solene, tal que Deus continua:

        \bBlockQuote{% !j -i12 116
            De todo aquele que não ouvir as minhas palavras, que ele falar em meu nome, disso  lhe  pedirei  contas.
            Porém o profeta que presumir de falar alguma palavra em meu nome, que eu lhe não mandei falar, ou o  que
            falar em nome de outros deuses, esse profeta será morto.}{Dt~18.19,20}{ARA}

    Aqui as implicações são seríssimas --- vida ou morte!  ---  Tal  que  se  torna  \emph{absolutamente  imperioso}  distinguir
    adequadamente a Palavra do Senhor daquela de falsos profetas.

    O texto segue, providencialmente, nesta exata direção: \bLineQuote{Se disseres no teu coração: Como conhecerei a palavra que
    o Senhor não falou?}{Dt~18.21}{ARA}, e a resposta divina \emph{não deixa dúvidas}:

        \bBlockQuote{% !j -i12 116
            \textbf{Sabe que}, quando esse profeta falar em nome do Senhor, e a palavra dele se \textbf{não cumprir,
            nem suceder, como profetizou}, esta é palavra que o Senhor \textbf{não disse}; com soberba,  a  falou  o
            tal profeta; não tenhas temor dele.}{Dt~18.22}{ARA}

    Este é um cenário de apenas duas possibilidades: ou a profecia (i)~é de Deus, ou ela (ii)~não é.  O  texto  sagrado  aqui  é
    \emph{suficiente} para a determinação de  todos  os  dois  possíveis  casos,  pelo  emprego  da  lógica  mais  elementar  no
    entendimento do texto. Se uma possibilidade foi enunciada, sua \emph{negação} leva, necessariamente, à outra.

    Desta forma, tem-se que \textbf{a palavra que o Senhor diz cumpre-se \textsc{como profetizada}}, de acordo com Dt~18.22!

    Elimina-se, efetivamente,  qualquer  possibilidade  de  interpretação  alegorizada,  diferente  de  como  está  escrito,  de
    \textbf{como foi profetizado}.

    Importa pontuar que a própria profecia do verso~18 cumpriu-se \textbf{\textsc{literalmente}} em Jesus Cristo:
 
        \bBlockQuote{% !j -i12 116
            Não crês que eu estou no Pai e que o Pai está em mim? \textbf{As palavras que eu vos digo  não  as  digo
            por mim mesmo; mas o Pai, que permanece em mim, faz as suas obras}.}{Jo~14.10}{ARA}

    Foi profetizado \bQuote{em cuja boca porei as minhas palavras}, e cumpriu-se \textbf{como profetizado}!

    E ainda, com relação ao que foi profetizado: \bQuote{ele lhes falará tudo o que eu lhe ordenar}, temos o registro do
    cumprimento, em Jesus Cristo, assim:

        \bBlockQuote{% !j -i12 116
            Então, lhes falou Jesus: Em verdade, em verdade vos digo que \textbf{o  Filho  nada  pode  fazer  de  si
            mesmo}, senão \textbf{somente} aquilo que vir fazer o Pai; porque \textbf{tudo o que este fizer, o Filho
            também semelhantemente o faz}.}{Jo~5.19}{ARA}

    E ainda:

        \bBlockQuote{% !j -i12 116
            E  aquele  que  me  enviou  está  comigo,  não  me  deixou  só,  porque  eu  faço  sempre  o   que   lhe
            agrada.}{Jo~8.29}{ARA}

    Assim, pelas Escrituras, \textbf{profecia de Deus cumpre-se como foi profetizada}.

    %---------------------------------------------------------------------------------------------------------------------------
    \subsection{Algumas Implicações}

    Há importantes implicações em se estudar profecia ``segundo Deus,'' conforme o  que  foi  resumidamente  estabelecido  pelas
    Escrituras neste estudo --- mas principalmente pelo que está estabelecido \textsc{de fato} nos céus,  no  coração  de  Deus,
    onde nenhum homem pode mal-intencionadamente intrometer-se!

    \begin{enumerate}

        \item Pelo princípio bíblico de \emph{unicidade}, por exemplo, servos de Jesus Cristo não deveriam tolerar a existência
            de múltiplas `teorias' proféticas ou `linhas de interpretação esactológicas' de um único livro sagrado!

        \item Pelos princípios bíblicos de \emph{veracidade de Deus}, da \emph{verificabilidade das profecias}, e de que
            \emph{profecia de Deus cumpre-se como profetizada}, linhas de interpretação alegóricas de profecias, que fogem do
            como está profetizado, jamais deveriam sequer ser consideradas, seja acadêmica ou devocionalmente. Pelo contrário,
            deveriam ser reprovadas e rejeitadas como pecado de rebelião contra o Senhor, nosso Deus e contra Sua Palavra!

    \end{enumerate}

    A retumbante falha nestes quesitos básicos faz com que a Teologia, em seu estado de coisas e visibilidade atual,  abrigue  e
    conviva com questões que em outras áreas do conhecimento, mais exatas, seriam consideradas absurdas, ridículas e  patéticas,
    a exemplo de propostas amilenistas de números não sendo descritivos, porém representativos  ---  tal  que  passagens,  como:
    \bLineQuote{e viveram e reinaram com Cristo durante mil anos}{Ap~20.4}{ARA} não signifiquem o que nelas  está  escrito!  ---
    frontalmente violando o que é ordenado nas Escrituras em Dt~18, como exposto acima.

    Ainda, as Escrituras exortam a que a igreja tenha um só pensamento, para a completa alegria: \bLineQuote{completai  a  minha
    alegria, de modo que  \emph{penseis  a  mesma  coisa},  tenhais  o  mesmo  amor,  sejais  unidos  de  alma,  tendo  o  mesmo
    sentimento.}{Fp~2.2}{ARA}.


% !center 124 | frame 124 -f'\%-\% '
%------------------------------------------------------------------------------------------------------------------------------%
%                                                A Tribulação Pelas Escrituras                                                 %
%------------------------------------------------------------------------------------------------------------------------------%

\section{A Tribulação Pelas Escrituras}

    Parágrafo.

%---------------------------------------------------------------------------------------------------------------------------
\subsection{A Tribulação na Lei}

    Parágrafo.

%---------------------------------------------------------------------------------------------------------------------------
\subsection{A Tribulação nos Escritos}

    Parágrafo.

%---------------------------------------------------------------------------------------------------------------------------
\subsection{A Tribulação nos Profetas}

    Parágrafo.


% !center 124 | frame 124 -f'\%-\% '
%------------------------------------------------------------------------------------------------------------------------------%
%                                                         Conclusions                                                          %
%------------------------------------------------------------------------------------------------------------------------------%

\section{Conclusão}

    Conclusão.

%-------------------------------------------------------------------------------------------------------------------------------
